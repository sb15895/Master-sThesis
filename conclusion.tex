\section{Conclusion}
\label{sec:conclusion}
This project aims to find the effects of changing the clamp widths on the load bearing capabilities of a deployable boom to accurately estimate the limits of the boom and thereby predict the buckling of the boom. This is necessary as the booms are attached to a cylindrical canister using two screws, and not the typical encastered condition in which the whole of root is assumed to be fixed. These conditions are modelled using various downwards displacement to obtain a fixed width on the boom. Changing the clamp widths has effects on the ploy length which is defined to be 5\% magnitude of the undeformed width of the boom and also the rotational moment on the tip. The clamp width is related to the ploy length with a second order relationship and the maximum rotational moment is related to the clamp width with a linear relationship. 
From the simulation it was found that the boom bends slightly inwards but to an almost constant extent for all the cases. The reasons for this are not yet clear but can be looked in the future. It would be interesting to note the effects of rotational moment in the opposite sense bending with varying clamp widths and compare the relationship with change in clamp width. An ABAQUS script could also be made in order to give more data points which can confirm these relationships. The increasing propagation moments with a decrease in clamp widths can also be explored. Another solution developed in reference \cite{Footdale2014} is a deployment mechanism with a clearance cut of the boom. This allows for a narrower width of a boom, however it would be useful to study the trade-offs of this method given that a narrower boom might result in a lower bending stiffness. 

